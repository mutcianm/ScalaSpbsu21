\documentclass[aspectratio=169]{beamer}

% SETUP =====================================
\usepackage[T1,T2A]{fontenc}
\usepackage[utf8]{inputenc}
\usepackage[russian]{babel}
\usepackage{listings}
\usepackage{array}
\usepackage{amssymb}
\usepackage{pifont}
\usepackage{minted}
\usepackage{../../beamerthemeslidesgeneric}
% SETUP =====================================

\title{Scala Library I}
\author{Mikhail Mutcianko, Alexey Shcherbakov}
\institute{СПБгУ, СП}
\date{11 марта 2021}

\begin{document}

\frame{\titlepage}

\section{Scala Library: Basic Collections}

% option
\begin{frame}{Option}
\texttt{Option[A]} is a container for an optional value of type \texttt{A}
\vspace{1em}
\begin{itemize}
  \item If the value of type \texttt{A} is present, the \texttt{Option[A]} is an instance of \texttt{Some[A]}
  \item If the value is absent, the \texttt{Option[A]} is the object \texttt{None}
\end{itemize}
\end{frame}

\begin{frame}[fragile]{Option}{monad}
\begin{minted}{scala}
  trait Monad[T] {
    def flatMap[U](f: T => Monad[U]): Monad[U]
  }

  def unit[T](x: T): Monad[T]
\end{minted}
\end{frame}

\begin{frame}{Option}
\begin{itemize}
  \item create with \texttt{Some(\ldots)} or \texttt{None}
  \item transform with \texttt{map} or \texttt{flatMap}
  \item test contents with \texttt{exists} or \texttt{contains}
  \item check kind with \texttt{isDefined} or \texttt{isEmpty}
  \item unwrap with \texttt{get} or \texttt{getOrElse}
\end{itemize}
\end{frame}

% tuple
\begin{frame}[fragile]{Tuple}
  Scala tuple combines a fixed number of items together so that they can be passed around as a
  whole.
  \begin{itemize}
    \item one-indexed
    \item immutable
    \item unlike an array or list, a tuple can hold objects with different types
    \item tuples are a syntactic sugar
  \end{itemize}
  \pause
\end{frame}

\begin{frame}[fragile]{Tuple}
\begin{minted}{scala}
val t = (1, "hello", Console)
val t = new Tuple3(1, "hello", Console) // same

val tuple = ("apple", "dog") // access by index
val fruit = tuple._1
val animal = tuple._2

val student = ("Sean Rogers", 21, 3.5)
val (name, age, gpa) = student // deconstruction

val tuple = ("apple", 3).swap // Tuple2 swap
\end{minted}
\end{frame}

\begin{frame}{Mutable and Immutable}
  \begin{itemize}
    \item Scala collections are immutable by default
    \item mutable collections have to be explicitly imported 
      \texttt{import scala.collection.mutable.Map}
    \item all transforming operations return a modified copy
    \item mutable collections can be modified by methods with \texttt{=} suffix, e.g. \texttt{++=}
    \item mind the variance: immutable[+T] $\leftrightarrow$ mutable[T]
  \end{itemize}
\end{frame}

% PICTURE - collections hierarchy
\begin{frame}[fragile]{Collections hierarchy}
  \begin{center}
    \hspace*{0.2\textwidth}
    \begin{forest}
      [Traversable, name=top
        [Iterable
          [Seq]
          [Set]
          [Map]
      ]]
      \node[right=of top,draw,dotted,label=below:{\tiny\color{black!50} methods come from here}]
        (gto){\footnotesize\ttfamily\color{black!50} GenTraversableOnce};
      \draw[->,dotted](gto) -- (top);
    \end{forest}
  \end{center}
  \pause
  \begin{itemize}
    \item \texttt{Traversable} trait lets you traverse an entire collection
    \item \texttt{Iterable} trait defines an iterator, which lets you loop through a collection’s
      elements one at a time
  \end{itemize}
\end{frame}

\begin{frame}{Traversable}{Methods}
\begin{itemize}
  \item concatenate collections with \texttt{++}
  \item transform elements with \texttt{map} \texttt{flatMap} and \texttt{collect} \ldots
  \item convert to other collections with \texttt{toList} / \texttt{toMap} / \texttt{toSet} \ldots
  \item check size with \texttt{isEmpty} / \texttt{nonEmpty} / \texttt{size}
  \item access elements with \texttt{head} / \texttt{last} / \texttt{find} \ldots
  \item narrow down with \texttt{filter} / \texttt{collect} / \texttt{take} \ldots
  \item split with \texttt{partition} / \texttt{groupBy} / \texttt{span} \ldots
  \item fold with \texttt{foldLeft} / \texttt{reduce} / \texttt{aggregate} \ldots
\end{itemize}
\end{frame}
% basic traits: Traversable Iterable etc
% list - alvin
\begin{frame}[fragile]{Sequences}
  \begin{center}
    \begin{forest}
    [Seq, s sep=4em
     [IndexedSeq, name=indseq
      [WrappedArray]
      [Range]
      [String]]
      [Buffer, l=9em, anchor=south
      [ArrayBuffer, name=arbuf] 
      [ListBuffer]]
     [LinearSeq
      [List]
      [Queue]
      [Stack]
      [Stream]]
    ]
    \draw[->,dotted] (indseq) -- (arbuf);
    \end{forest}
  \end{center}
\end{frame}

\begin{frame}{Sequences}
  \begin{itemize}
    \item \texttt{IndexedSeq} --- indicates that random access of elements is efficient
    \item \texttt{LinearSeq} --- can be efficiently split into head and tail components
    \item \texttt{Buffer} --- efficient appending, prepending, or inserting new elements
  \end{itemize}
\end{frame}

% map - alvin
\begin{frame}[fragile]{Map}
  \tikzset{every node/.style={
    font=\scriptsize\ttfamily,
    minimum height=2em,minimum width=6em
  }}
\begin{center}
\begin{tikzpicture}[scale=1, transform shape]
  \node[draw](map){Map};
  \node[draw,below=of map](sortedmap){SortedMap};
  \node[draw,dotted,right=of map](multi){MultiMap};
  \node[draw,left=of sortedmap](hashmap){HashMap};
  \node[draw,right=of sortedmap](listmap){ListMap};
  \node[draw,below=of sortedmap](weakmap){WeakHashMap};
  \node[draw,left=of weakmap](treemap){TreeMap};
  \node[draw,right=of weakmap](lhashmap){LinkedHashMap};
  \draw[->] (map) -- (sortedmap);
  \draw[->] (map) -- (hashmap);
  \draw[->] (map) -- (listmap);
  \draw[->] (map) -- (treemap);
  \draw[->] (map) -- (weakmap.east);
  \draw[->] (map) -- (lhashmap);
\end{tikzpicture}
\end{center}
\end{frame}

\begin{frame}{Map}
  \begin{itemize}
    \item is an \texttt{Iterable} consisting of pairs of keys and values
    \item can be constructed from a \texttt{Seq[Tuple2[\_,\_]]}: \\ \texttt{Seq(1 -> "foo").toMap}
    \item \texttt{apply} throws exeption when key isn't found
    \item \texttt{get} returns an \texttt{Option}
    \item access subcollections via \texttt{keys} and \texttt{values}
  \end{itemize}
\end{frame}

\begin{frame}[fragile]{Map}{example}
\begin{minted}{scala}
val m = Map(1 -> "a", 2 -> "b")
val newMap = m + (3 -> "foo")

m.contains(3) == false
m(3) // NoSuchElementException
m.get(3).isDefined == false
\end{minted}
\end{frame}

% set - alvin
\begin{frame}[fragile]{Set}
  \tikzset{every node/.style={
    draw,
    font=\scriptsize\ttfamily,
    minimum height=2em,minimum width=6em
  }}
\begin{center}
    \begin{tikzpicture}
    \node (set) {Set};
    \node[below left= 2em and -4em of set] (hashset) {HashSet};
    \node[below right= 2em and 0em of set] (listset) {ListSet};
    \node[left= of hashset] (bitset) {BitSet};
    \node[right= of listset] (sortedset) {SortedSet};
    \node[below= of sortedset] (treeset) {TreeSet};
    \draw[->] (set) -- (listset);
    \draw[->] (set) -- (hashset);
    \draw[->] (set) -- (bitset);
    \draw[->] (set) -- (sortedset);
    \draw[->] (sortedset) -- (treeset);
  \end{tikzpicture}
\end{center}
\end{frame}

\begin{frame}{Set}
Sets are \texttt{Iterables} that contain no duplicate elements
\vspace{1em}
\begin{itemize}
  \item test: \texttt{contains} / \texttt{apply} / \texttt{subsetOf}
  \item add: \texttt{+} / \texttt{++}
  \item remove: \texttt{-} / \texttt{-{}-}
  \item operations: \texttt{intersect} / \texttt{union} / \texttt{diff}
\end{itemize}
\end{frame}

\begin{frame}{Predef}
 \begin{block}{}
   The \texttt{Predef} object provides definitions that are accessible in all Scala compilation units without
   explicit qualification
 \end{block}
 \vspace{1em}
 \pause
 \begin{itemize}
   \item methods: \texttt{??? assert print classOf \ldots}
   \item aliases: \texttt{Class Function Map Set String -> \ldots}
   \item implicits: \texttt{Long2long long2Long genericArrayOps intArrayOps wrapString \ldots} 
 \end{itemize}
\end{frame}

%for comprehentions - t slides

\section{Java collections}

\begin{frame}{Scala vs Java vs Kotlin}
  \begin{itemize}
    \item Java's collections are a part of JRE
      \pause
    \item Kotin library re-uses Java collections by adding extension methods
      \pause
    \item Scala implememnts own collections(except for \texttt{Array})
      \pause
      \item Java $\leftrightarrow$ Scala ?
  \end{itemize}  
\end{frame}

\begin{frame}[fragile]{Conversion}{Two-way}
\begin{verbatim}
Iterator               <=>     java.util.Iterator
Iterator               <=>     java.util.Enumeration
Iterable               <=>     java.lang.Iterable
Iterable               <=>     java.util.Collection
mutable.Buffer         <=>     java.util.List
mutable.Set            <=>     java.util.Set
mutable.Map            <=>     java.util.Map
mutable.ConcurrentMap  <=>     java.util.concurrent.ConcurrentMap
\end{verbatim}
\end{frame}

\begin{frame}[fragile]{Conversion}{One-way}
\begin{verbatim}
Seq           =>    java.util.List
mutable.Seq   =>    java.util.List
Set           =>    java.util.Set
Map           =>    java.util.Map
\end{verbatim}
\end{frame}

\begin{frame}[fragile]{collection.JavaConverters}
  \begin{itemize}
    \item \mintinline{scala}{import collection.JavaConverters._} \cite{javaconv}
    \item  conversions work by setting up a “wrapper” object that forwards all operations to the
      underlying collection object
    \item collections are never copied when converting between Java and Scala
    \item attempting to mutate immutabe collection from java will yield
      \texttt{UnsupportedOperationException}
  \end{itemize}
\end{frame}

\begin{frame}[fragile]{collection.JavaConverters}{Example}
\begin{minted}{scala}
import collection.mutable._
import collection.JavaConverters._

val jul: java.util.List[Int] = ArrayBuffer(1, 2, 3).asJava
val buf: Seq[Int] = jul.asScala
val m: java.util.Map[String, Int] = HashMap("abc" -> 1, "hello" -> 2).asJava

val jul = List(1, 2, 3).asJava
jul.add(7) // throws UnsupportedOperationException
\end{minted}
\end{frame}

\section{Scala Library: Best and Worst Practices}

\begin{frame}[fragile]{Guidelines}
  Basic usage rules for Scala collections \cite{twitter1}
  \pause
  \begin{itemize}
    \item prefer using immutable collections
      \pause
    \item use the mutable namespace explicitly: \\
      \mintinline{scala}{val set = mutable.Set()}
      \pause
    \item use the default constructor for the collection type: \\
      \mintinline{scala}{val seq = Seq(1, 2, 3)}
      \pause
    \item prefer default collections over specific ones
  \end{itemize}
\end{frame}

\begin{frame}[fragile]{Sequences}
  \begin{itemize}
    \item Prefer length to size for arrays\\
      \texttt{Array.size} calls are still implemented via implicit conversion
      \pause
    \item Create empty collections explicitly: \\
      \begin{minted}{scala}
      Seq[T]()        |\xmark|
      Seq.empty[T]    |\cmark|
      \end{minted}
      \pause
    \item Don’t negate emptiness-related properties \\
      \begin{minted}{scala}
      !seq.isEmpty    |\xmark|
      seq.nonEmpty    |\cmark|
      \end{minted}
      \pause
    \item Don’t compute length for emptiness check \\
      \begin{minted}{scala}
      seq.length > 0  |\xmark|
      seq.nonEmpty    |\cmark|
      \end{minted}
  \end{itemize}
\end{frame}

\begin{frame}[fragile]{Eqality}
  \begin{itemize}
    \item Don’t rely on == to compare array contents \\
      \begin{minted}{scala}
      array1 == array2              |\xmark|
      array1.sameElements(array2)   |\cmark|
      \end{minted}
      \pause
    \item Don’t check equality between collections in different categories \\
      \begin{minted}{scala}
      seq == set        |\xmark|
      seq.toSet == set  |\cmark|
      \end{minted}
  \end{itemize}
\end{frame}

\begin{frame}[fragile]{Indexing}
  \begin{itemize}
    \item use \texttt{head} and \texttt{last} instead of index access
      \pause
    \item use \texttt{headOption} and \texttt{lastOption} instead of bound checks
      \pause
    \item use \texttt{indices} instead of manual \texttt{Range} construction
  \end{itemize}
\end{frame}

\begin{frame}[fragile]{Existence}
  \begin{itemize}
    \item Don’t use equality predicate to check element presence \\
      \begin{minted}{scala}
      seq.exists(_ == x)  |\xmark|
      seq.contains(x)     |\cmark|
      \end{minted}
      \pause
    \item Use exists for everything else \\
      \begin{minted}{scala}
      seq.count(p) > 0        |\xmark|
      seq.filter(p).nonEmpty  |\xmark|
      seq.find(p).isDefined   |\xmark|

      seq.exists(p)           |\cmark|
      \end{minted}
  \end{itemize} 
\end{frame}

\begin{frame}[fragile]{Filtering}
  \begin{itemize}
    \item Don’t negate filter predicate\\
      \begin{minted}{scala}
      seq.filter(!p)     |\xmark|
      seq.filterNot(p)   |\cmark|
      \end{minted}
      \pause
    \item Don’t resort to filtering to count elements\\
      \begin{minted}{scala}
      seq.filter(p).length  |\xmark|
      seq.count(p)          |\cmark|
      \end{minted}
      \pause
    \item Don’t use filtering to find first occurrence\\
      \begin{minted}{scala}
      seq.filter(p).headOption  |\xmark|
      seq.find(p)               |\cmark|
      \end{minted}
  \end{itemize}
\end{frame}

\begin{frame}[fragile]{Sorting}
  \begin{itemize}
    \item Don’t sort by a property manually\\
      \begin{minted}{scala}
      seq.sortWith(_.property <  _.property) |\xmark|
      seq.sortBy(_.property)                 |\cmark|
      \end{minted}
      \pause
    \item Perform reverse sorting in one step\\
      \begin{minted}{scala}
      seq.sorted.reverse               |\xmark|
      seq.sorted(Ordering[T].reverse)  |\cmark|
      \end{minted}
      \pause
    \item Don’t use sorting to find the smallest or biggest element\\
      \begin{minted}{scala}
      seq.sorted.reverse.head      |\xmark|
      seq.sortBy(_.property).head  |\xmark|
      seq.max                      |\cmark|
      seq.minBy(_.property)        |\cmark|
      \end{minted}
  \end{itemize}
\end{frame}

\begin{frame}{TBC}
  \centering\Large More on Paval Fatin's blog \cite{pasha-collections}
\end{frame}

\begin{thebibliography}{9}
  \bibitem{pasha-collections}
  \url{https://pavelfatin.com/scala-collections-tips-and-tricks/}
  \bibitem{javaconv}
  \url{https://docs.scala-lang.org/overviews/collections/conversions-between-java-and-scala-collections.html}
  \bibitem{twitter1}
  \url{https://twitter.github.io/effectivescala/#Collections-Use}
\end{thebibliography}

\end{document}

