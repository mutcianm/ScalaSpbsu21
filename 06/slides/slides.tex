\documentclass[aspectratio=169]{beamer}

% SETUP =====================================
\usepackage[T1,T2A]{fontenc}
\usepackage[utf8]{inputenc}
\usepackage[russian]{babel}
\usepackage{listings}
\usepackage{array}
\usepackage{amssymb}
\usepackage{pifont}
\usepackage{minted}
\usepackage{../../beamerthemeslidesgeneric}
% SETUP =====================================

\title{Scala Type System III}
\author{Mikhail Mutcianko, Alexey Shcherbakov}
\institute{СПБгУ, СП}
\date{11 марта 2021}

\begin{document}

\frame{\titlepage}

\section{Structural Types}

\begin{frame}{Disclaimer}
  \begin{center}
    \LARGE \alert{Do not} use structural types unless you're 100\% sure you need them
  \end{center}
\end{frame}

\begin{frame}[fragile]{Structural Types}
  \begin{block}{}
    \textit{Structural type} is a type whose conformance is defined by its strucrure(a set of
    contained signatures) only. 
  \end{block}
  \bigskip
  \pause
  \begin{minted}{scala}
  type withFoo = { def foo(x: Int): String }

  def foo(x: withFoo)
  def bar(x: { def foo(x: Int): String })
  \end{minted}
\end{frame}

\begin{frame}{Structural Types}
  \begin{itemize}
    \item implemented using runtime reflection --- watch performance
    \item \textit{usecase}: combine with \texttt{.asInstanceOf} for straightforward reflective method
      calls
  \end{itemize}
\end{frame}

\section{This Types}
\begin{frame}[fragile]{Singleton types}
  \begin{block}{}
    \textit{Singleton type} is a type of a single value, to which only this particular value and
    \texttt{null} conform.
  \end{block}
  \pause
  \begin{minted}{scala}
  val hello = "Hello"
  
  val notHello:  hello.type = "world" // error: type mismatch
  val alsoHello: hello.type = hello   // OK

  |\pause|object A
  def method(): A.type = A
  \end{minted}
\end{frame}

\begin{frame}[fragile]{Singleton Types}{Method chaining}
\begin{minted}{scala}
class A 
  { def method1: this.type = { ...; this } }
class B extends A 
  { def method2: this.type = { ...; this } }

new B.method1.method2 // OK
\end{minted}
\end{frame}

\section{F-bound Polymorphism}
\begin{frame}{Problem}
  \begin{block}{Frequent question}
      I have a type hierarchy … how do I declare a supertype method that returns the “current”
      type?\cite{f-bounds}
  \end{block}
\end{frame}

\begin{frame}[fragile]{F-bound Polymorphism}
\begin{onlyenv}<1>
\begin{minted}{scala}
trait Pet { 
  def renamed: Pet 
}

abstract class Dog extends Pet

class Cat extends Pet {
  override def renamed: Dog = ??? // OOPS
}
\end{minted}
\end{onlyenv}

\begin{onlyenv}<2>
\begin{minted}{scala}
trait Pet {
  type T <: Pet
  def renamed: T 
}

abstract class Dog extends Pet

class Cat extends Pet {
  override type T = Dog
  override def renamed: Dog = ??? // still OOPS
}
\end{minted}
\end{onlyenv}
\begin{onlyenv}<3>
\begin{minted}{scala}
trait Pet {
  type T >: this.type <: Pet
  def renamed: T 
}

abstract class Dog extends Pet

class Cat extends Pet {
  override type T = Dog   // error: type T has incompatible type
  override def renamed: Dog = ???
}
\end{minted}
\end{onlyenv}
\begin{onlyenv}<4>
\begin{minted}{scala}
trait Pet {
  type T >: this.type <: Pet
  def renamed: T 
}

abstract class Dog extends Pet

class Cat extends Pet {
  override type T = Cat
  override def renamed: Cat = ??? // OK |\cmark|
}
\end{minted}
\end{onlyenv}
\end{frame}

\section{Existential Types}
\begin{frame}{Existential Types}
  \begin{block}{}
    \textit{Existential type} lets up assume \textbf{existence} of some type parameter without
    exposing or providing it explicitly\cite{exis-adv}
  \end{block}
\end{frame}

\begin{frame}[fragile]{Existential Types}
  \begin{onlyenv}<1>
    \begin{minted}{scala}
    trait Foo[T] { def foo: T }
    
    type Dep[A] = Foo[A]                    // A is required on lhs
    type Exis   = Foo[A] forSome { type A } // A only in rhs
    type ExisA  = Foo[_]                    // A is not required at all
    
    def bar(x: Exis) = x.foo // return type ?
    \end{minted}
  \end{onlyenv}
\begin{onlyenv}<2>
    \begin{minted}{scala}
    trait Foo[T] { def foo: T }
    
    type Dep[A] = Foo[A]                           // A is required on lhs
    type Exis   = Foo[A] forSome { type A <: Int } // A only in rhs
    type ExisA  = Foo[_ <: Int]                    // A is not required at all
    
    def bar(x: Exis): Int = x.foo
    \end{minted}
  \end{onlyenv}
\end{frame}

\section{Higher-Kinded Types}

\begin{frame}{Kinds}
  \begin{block}{}
    \textit{Kinds} are an abstraction over types that implement parametric polymorphism in Scala
  \end{block}
  \bigskip
  Tere are two \textit{kinds} in Scala:
  \begin{itemize}
    \item \textbf{*} --- a proper type
    \item $\rightarrow$ --- a type constructor
  \end{itemize}
\end{frame}

\begin{frame}[fragile]{Kinds}{Example}
\begin{verbatim}
scala> :kind -v String
String's kind is A { * }
This is a proper type.

scala> :kind -v List
List's kind is F[+A] { * -(+)-> * }
This is a type constructor: a 1st-order-kinded type.
\end{verbatim}
\end{frame}

\begin{frame}[fragile]{Type Constructor}
  \begin{block}{}
    A \textit{type constructor} is a type-level function of a $\rightarrow$ kind, that takes type
    parameters and applies them to yield a proper type or another type constructor
  \end{block}
  \bigskip
  \begin{minted}{scala}
  type IntList = List[Int]
  type StringOrInt = Either[String, Int]

  type StringOr??? = Either[String, _] // how?
  \end{minted}
\end{frame}

\begin{frame}{Higher Kinded Types}
  \begin{block}{}
    A \textit{higher kinded type} is a type constructor that accepts or returns another type constructor.\\
    This is analogous to a higher-order function, and denoted as $(* \rightarrow *) \rightarrow *$.
  \end{block}
\end{frame}

\begin{frame}[fragile]{Higher Kinded Types}{Example}
\begin{minted}{scala}
 trait Functor[F[_]] {
   def map[A, B](fa: F[A])(f: A => B): F[B]
 }

 val x: Functor[Option] = ???
 x.map(Some(123)) { x => x.toString } // :Option[String]
\end{minted}
\end{frame}

\subsection{Partial Application of HKT}
\begin{frame}[fragile]{Partial Application of HKT}
  \begin{block}{}
    If higher kinded types behave similarly to higher order functions, can they also be partially
    applied?
  \end{block}
  \bigskip
  \begin{minted}{scala}
  type StringOr = Either[_, String] // this is an existential type
  \end{minted}
\end{frame}

\begin{frame}[fragile]{Partial Application of HKT}
\begin{onlyenv}<1>
\begin{minted}{scala}
val efs: Functor[Either[_, String]] = ??? 
              // ^ error: Either[_, String] takes no type parameters, expected: one
\end{minted}
\end{onlyenv}
\begin{onlyenv}<2>
\begin{minted}{scala}
val efs = new Functor[???] {
    override def map[A, B](fa: Either[Int, A])(f: A => B): Either[Int, B] = ???
}
\end{minted}
\end{onlyenv}
\begin{onlyenv}<3>
\begin{minted}{scala}
type IntOrA[A] = Either[Int, A]
val efs = new Functor[IntOrA] {
    override def map[A, B](fa: Either[Int, A])(f: A => B): Either[Int, B] = ???
}
\end{minted}
\end{onlyenv}
\begin{onlyenv}<4>
\begin{minted}{scala}
val efs = new Functor[({type lam[Y] = Either[Int, Y]})#lam] {
    override def map[A, B](fa: Either[Int, A])(f: A => B): Either[Int, B] = ???
}
\end{minted}
\end{onlyenv}
\end{frame}

\begin{frame}[fragile]{Kind Projector}
  Type lambda syntax in Scala 2 is verbose. Kind projector plugin\cite{kind-proj} solves this issue
  for HKT-heavy applications.
  \bigskip
  \pause
  \begin{minted}{scala}
    Tuple2[*, Double]        // equivalent to: type R[A] = Tuple2[A, Double]
    Either[Int, +*]          // equivalent to: type R[+A] = Either[Int, A]
    Function2[-*, Long, +*]  // equivalent to: type R[-A, +B] = Function2[A, Long, B]
    EitherT[*[_], Int, *]    // equivalent to: type R[F[_], B] = EitherT[F, Int, B]
  \end{minted}
\end{frame}

\section{AUX Pattern}
\begin{frame}{AUX Pattern}
  \begin{block}{scalac error}
    illegal dependent method type: parameter appears in the type of another parameter in the same
    section or an earlier one
  \end{block}
  \bigskip
  Scala tells us that we can’t use the dependent type in the same section, we can use it in the
  \textbf{next} parameters block or as a \textbf{return type} only.

  This is where AUX pattern\cite{aux} comes into play
\end{frame}

\begin{frame}[fragile]{AUX Pattern}
  \begin{onlyenv}<1>
  \begin{minted}{scala}
  trait Foo[A] {
    type B
    def value: B
  }
  
  def foo[T](t: T)(implicit f: Foo[T]): f.B = f.value                // OK
  def foo[T](t: T)(implicit f: Foo[T], m: Monoid[f.B]): f.B = m.zero // error
  \end{minted}
  \end{onlyenv}
\begin{onlyenv}<2>
  \hig{7}
  \begin{minted}{scala}
  trait Foo[A] {
    type B
    def value: B
  }
  
  def foo[T](t: T)(implicit f: Foo[T]): f.B = f.value
  def foo[T, B0](t: T)(implicit f: Foo[T] {type B = B0}, m: Monoid[B0]): f.B = m.zero // OK
  \end{minted}
  \end{onlyenv}
  \begin{onlyenv}<3>
  \begin{minted}{scala}
  trait Foo[A] {
    type B
    def value: B
  }
  object Foo {
    type Aux[A0, B0] = Foo[A0] { type B = B0  }
  }

  def foo[T, R](t: T)(implicit f: Foo.Aux[T, R], m: Monoid[R]): R = m.zero 
  \end{minted} 
  \end{onlyenv}
\end{frame}

\section{Magnet Pattern}
\begin{frame}[fragile]{Magnet Pattern}
  \begin{block}{Problem statement}
    How to define multiple method overloads with only difference in parametrized types?
  \end{block}
  \bigskip
  \pause
  \begin{minted}{scala}
  def complete(future: Future[HttpResponse]): Int
  def complete(future: Future[StatusCode]): Int
  \end{minted}
\end{frame}

\begin{frame}[fragile]{Magnet Pattern}
Define a magnet trait:
\begin{minted}{scala}
sealed trait FutureMagnet {
  type Result
  def apply() : Result
}

def completeFuture(magnet: FutureMagnet):magnet.Result = magnet()
\end{minted}
\end{frame}

\begin{frame}[fragile]{Magnet Pattern}
\begin{minted}{scala}
implicit def fromHttpResponseFuture(future: Future[HttpResponse]) =
  new CompletionMagnet {
    type Result = Int
    def apply(): Result = ... // implementation using future
  }
implicit def fromStatusCodeFuture(future: Future[StatusCode]) =
  new CompletionMagnet {
    type Result = Int
    def apply(): Result = ... // implementation using future
  }
\end{minted}
\end{frame}

\begin{thebibliography}{9} 
\bibitem{exis-adv}
  \url{https://pjrt.medium.com/existential-types-in-scala-6321f19c4a57}
\bibitem{f-bounds}
  \url{https://tpolecat.github.io/2015/04/29/f-bounds.html}
\bibitem{kubuszok}
  \url{https://kubuszok.com/compiled/kinds-of-types-in-scala}
\bibitem{kind-proj}
  \url{https://github.com/typelevel/kind-projector}
\bibitem{aux}
  \url{https://gigiigig.github.io/posts/2015/09/13/aux-pattern.html}
\bibitem{magnet}
  \url{http://blog.madhukaraphatak.com/scala-magnet-pattern/}
\end{thebibliography}

\end{document}

